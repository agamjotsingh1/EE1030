%iffalse
\let\negmedspace\undefined
\let\negthickspace\undefined
\documentclass[journal,12pt,twocolumn]{IEEEtran}
\usepackage{cite}
\usepackage{amsmath,amssymb,amsfonts,amsthm}
\usepackage{algorithmic}
\usepackage{graphicx}
\usepackage{textcomp}
\usepackage{xcolor}
\usepackage{txfonts}
\usepackage{listings}
\usepackage{enumitem}
\usepackage{mathtools}
\usepackage{gensymb}
\usepackage{comment}
\usepackage[breaklinks=true]{hyperref}
\usepackage{tkz-euclide} 
\usepackage{listings}
\usepackage{gvv}                                        
%\def\inputGnumericTable{}                                 
\usepackage[latin1]{inputenc}                                
\usepackage{color}                                            
\usepackage{array}                                            
\usepackage{longtable}                                       
\usepackage{calc}                                             
\usepackage{multirow}                                         
\usepackage{hhline}                                           
\usepackage{ifthen}                                           
\usepackage{lscape}
\usepackage{tabularx}
\usepackage{array}
\usepackage{float}
\usepackage{circuitikz}

\newtheorem{theorem}{Theorem}[section]
\newtheorem{problem}{Problem}
\newtheorem{proposition}{Proposition}[section]
\newtheorem{lemma}{Lemma}[section]
\newtheorem{corollary}[theorem]{Corollary}
\newtheorem{example}{Example}[section]
\newtheorem{definition}[problem]{Definition}
\newcommand{\BEQA}{\begin{eqnarray}}
\newcommand{\EEQA}{\end{eqnarray}}
\newcommand{\define}{\stackrel{\triangle}{=}}
\theoremstyle{remark}
\newtheorem{rem}{Remark}

% Marks the beginning of the document
\begin{document}
\bibliographystyle{IEEEtran}
\vspace{3cm}

\renewcommand{\thefigure}{\theenumi}
\renewcommand{\thetable}{\theenumi}


\begin{circuitikz}

\draw (8,0) to [resistor, l_ = $1 \Omega$, color=cyan] (4,0) to [battery1, l=$80 V$, color=cyan] (0,0);

\draw (8,0) to [resistor, l_ = $20 \Omega$, color=cyan] (8,2);
\draw (6,2) to [battery1, l_= $45 V$, color=cyan] (8,2);
\draw (6,2) to [resistor, l_ = $1 \Omega$, color=cyan] (4,2) to [resistor, l_ = $40 \Omega$, color=cyan] (1,2) to [short](0,2);

\draw (0,0) to [short] (0,4);
\draw (0,4) to [resistor, l = $30 \Omega$, color=cyan](6,4);
\draw (6,4) to [short](8,4);
\draw (8,4) to [short](8,2);

% Writing nodes
\draw (0,2) node[label={left:a}, circ, color=cyan]{};
\draw (3.75,2) node[label={above:b}, circ, color=cyan]{};
\draw (6.5,2) node[label={above:c}, circ, color=cyan]{};
\draw (8,2) node[label={right:d}, circ, color=cyan]{};
\draw (0,2) node[label={left:a}, circ, color=cyan]{};
\draw (8,0) node[label={right:e}]{};
\draw (5,0) node[label={below:f}]{};
\draw (1.5,0) node[label={below:g}]{};
\draw (6,4) node[label={above:h}, circ, color=cyan]{};
\draw[->,shift={(4,3)},color=cyan] (120:.7cm) arc (120:-90:.7cm) node at(0,0){$I_1$};
\draw[->,shift={(4,1)},color=cyan] (120:.7cm) arc (120:-90:.7cm) node at(0,0){$I_2$};

\end{circuitikz}

\end{document}
