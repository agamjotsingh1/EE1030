\let\negmedspace\undefined
\let\negthickspace\undefined
\documentclass[journal]{IEEEtran}
\usepackage[a5paper, margin=10mm, onecolumn]{geometry}
%\usepackage{lmodern} % Ensure lmodern is loaded for pdflatex
\usepackage{tfrupee} % Include tfrupee package

\setlength{\headheight}{1cm} % Set the height of the header box
\setlength{\headsep}{0mm}     % Set the distance between the header box and the top of the text

\usepackage{gvv-book}
\usepackage{gvv}
\usepackage{cite}
\usepackage{amsmath,amssymb,amsfonts,amsthm,mathtools}
\usepackage{algorithmic}
\usepackage{graphicx}
\usepackage{textcomp}
\usepackage{xcolor}
\usepackage{txfonts}
\usepackage{listings}
\usepackage{enumitem}
\usepackage{mathtools}
\usepackage{gensymb}
\usepackage{comment}
\usepackage[breaklinks=true]{hyperref}
\usepackage{tkz-euclide} 
\usepackage{listings}
\def\inputGnumericTable{}                                 
\usepackage[latin1]{inputenc}                                
\usepackage{color}                                            
\usepackage{array}                                            
\usepackage{longtable}                                       
\usepackage{calc}                                             
\usepackage{multirow}                                         
\usepackage{hhline}                                           
\usepackage{ifthen}                                           
\usepackage{lscape}
\begin{document}

\bibliographystyle{IEEEtran}
\vspace{3cm}

\title{9.2.35}
\author{EE24BTECH11002 - Agamjot Singh}
% \maketitle
% \newpage
% \bigskip
{\let\newpage\relax\maketitle}
%\maketitle

\renewcommand{\thefigure}{\theenumi}
\renewcommand{\thetable}{\theenumi}
\setlength{\intextsep}{10pt} % Space between text and floats

\textbf{Question:}
\newline
Sketch the region $\brak{x, 0} \colon y = \sqrt{4 - x^2}$ and $x$-axis. Find the area of the region using integration.
\newline
\textbf{Solution:}

\begin{table}[h!]
	\centering
	\begin{tabular}[12pt]{ |c| c|}
    \hline
    \textbf{Variable} & \textbf{Description}\\ 
    \hline
	$\vec{A}$ & $(0, 0)$ point\\
    \hline
\end{tabular}
	\caption{Variables Used}
	\label{tab9.2.35}
\end{table}

The general equation of a circle is given by 
\begin{align}
	\norm{\vec{x}}^2 + 2\vec{u}^\top\vec{x} + f = 0
\end{align}
The given curve $y = \sqrt{4 - x^2}$ is that of a semicircle, since $y \geq 0$.
\newline
The equation of the curve can be written as
\begin{align}
	x^2 + y^2 - 4 = 0, y \geq 0
\end{align}
The parameters of the circle are
\begin{align}
	\vec{u} = \myvec{0\\0}, f = -4 \implies r = 2, \vec{O} = \myvec{0\\0}
\end{align}
Boundary of $D$ is the semicircle of radius $r$, which we can parameterize \brak{\text{in counter clock-wise orientation}} using
\begin{align}
	\myvec{x\\y} = \myvec{r\cos{t}\\r\sin{t}}, 0\geq t \leq \pi
\end{align}

By Green's Theorem,
\begin{align}
	\text{area of D}  = A &= \int \int \, dA\\
	&= \frac{1}{2} \int_{C} x\, dy - y\, dx\\
	&= \frac{1}{2} \int_{0}^{\pi} r^2\brak{\cos^2{t} + \sin^2{t}} \, dt\\
	&= \frac{r^2}{2} \int_{0}^{\pi} \, dt\\
	&= \frac{\pi r^2}{2}\\
	&= 2\pi
\end{align}

\begin{figure}[h!]
   \centering
   \includegraphics[width=0.7\linewidth]{figs/graph.png}
   \caption{Shaded area representing area of region given}
\end{figure}

\end{document}
