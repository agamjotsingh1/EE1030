\let\negmedspace\undefined
\let\negthickspace\undefined
\documentclass[journal,12pt,twocolumn]{IEEEtran}
\usepackage{cite}
\usepackage{amsmath,amssymb,amsfonts,amsthm}
\usepackage{algorithmic}
\usepackage{graphicx}
\usepackage{textcomp}
\usepackage{xcolor}
\usepackage{txfonts}
\usepackage{listings}
\usepackage{enumitem}
\usepackage{mathtools}
\usepackage{gensymb}
\usepackage{comment}
\usepackage[breaklinks=true]{hyperref}
\usepackage{tkz-euclide} 
\usepackage{listings}
\usepackage{gvv}                                        
%\def\inputGnumericTable{}                         
\usepackage[latin1]{inputenc}                                
\usepackage{color}                                            
\usepackage{array}                                            
\usepackage{longtable}                                       
\usepackage{calc}                                             
\usepackage{multirow}                                         
\usepackage{hhline}                                           
\usepackage{ifthen}                                           
\usepackage{lscape}
\usepackage{tabularx}
\usepackage{array}
\usepackage{float}
\usepackage{multicol}

\newtheorem{theorem}{Theorem}[section]
\newtheorem{problem}{Problem}
\newtheorem{proposition}{Proposition}[section]
\newtheorem{lemma}{Lemma}[section]
\newtheorem{corollary}[theorem]{Corollary}
\newtheorem{example}{Example}[section]
\newtheorem{definition}[problem]{Definition}
\newcommand{\BEQA}{\begin{eqnarray}}
\newcommand{\EEQA}{\end{eqnarray}}
\newcommand{\define}{\stackrel{\triangle}{=}}
\theoremstyle{remark}
\newtheorem{rem}{Remark}

\begin{document}
\bibliographystyle{IEEEtran}
\title{Assignment 1}
\author{EE24BTECH11002 - Agamjot Singh$^{*}$% <-this % stops a space
}
\maketitle
\newpage
\bigskip

\begin{enumerate}

\section*{\color{black}\colorbox{gray}{ C. }\color{white}\colorbox{magenta}{MCQs with One Correct Answer}}
    \setcounter{enumi}{4}

    %Question5
    \item The general solution of the trigonometric equation $\sin {x} + \cos{x} = 1$ is given by:
        
        \rightline{\color{magenta}\brak{1981 - 2 Marks}}
        \begin{enumerate}[label={(\alph*)}]
            \item $x = 2n\pi;\;n = 0,\pm1,\pm2\;...$
            \item $x = 2n\pi+\frac{\pi}{2};\;n = 0,\pm1,\pm2\;...$
            \item $x = n\pi+\brak{-1}^n\frac{\pi}{4}-\frac{\pi}{4};\;n = 0,\pm1,\pm2\;...$
            \item none of these
        \end{enumerate}

    %Question6
    \item The value of the expression $\sqrt{3}\;cosec\;20^\circ-sec\;20^\circ$ is equal to
        
        \rightline{\color{magenta}\brak{1988 - 2 Marks}}
        \begin{enumerate}[label={(\alph*)}]
            \begin{multicols}{2}
                \item 2
                \columnbreak
                \item $2\frac{\sin{20^\circ}}{\sin{40^\circ}}$
            \end{multicols}
            \begin{multicols}{2}
                \item 4
                \columnbreak
                \item $4\frac{\sin{20^\circ}}{\sin{40^\circ}}$
            \end{multicols}
        \end{enumerate}
        
    %Question7
    \item The general solution of \\$\sin{x}-3\sin{2x} + \sin{3x} = \cos{x}-3\cos{2x} + \cos{3x}$
        
        \rightline{\color{magenta}\brak{1989 - 2 Marks}}
        \begin{enumerate}[label={(\alph*)}]
            \begin{multicols}{2}
                \item $n\pi+\frac{\pi}{8}$
                \columnbreak
                \item $\frac{n\pi}{2}+\frac{\pi}{8}$
            \end{multicols}
            \begin{multicols}{2}
                \item $\brak{-1}^n\frac{n\pi}{2}+\frac{\pi}{8}$
                \columnbreak
                \item $2n\pi+\cos^{-1}{\frac{3}{2}}$
            \end{multicols}
        \end{enumerate}


    %Question8
    \item The equation $\brak{\cos{p}-1}x^2+\brak{\cos{p}}x+\sin{p}=0$ in the variable $x$, has real roots. Then $p$ can take any value in the interval
        
        \rightline{\color{magenta}\brak{1990 - 2 Marks}}
        \begin{enumerate}[label={(\alph*)}]
            \begin{multicols}{2}
                \item $\brak{0,2\pi}$
                \columnbreak
                \item $\brak{-\pi,0}$
                
            \end{multicols}

            \begin{multicols}{2}
                \item $\brak{-\frac{\pi}{2},\frac{\pi}{2}}$  
                \columnbreak
                \item $\brak{0,\pi}$
            \end{multicols}
        \end{enumerate}

    %Question9
    \item Number of solutions of the equation \\$\tan{x}+\sec{x} = 2\cos{x}$ lying in the interval $\brak{0, 2\pi}$ is
        
        \rightline{\color{magenta}\brak{1993 - 1 Marks}}
        \begin{enumerate}[label={(\alph*)}]
            \begin{multicols}{4}
                \item $0$
                \columnbreak
                \item $1$
                \columnbreak
                \item $2$
                \columnbreak
                \item $3$
            \end{multicols}
        \end{enumerate}

    %Question10
    \item Let $0<x<\frac{\pi}{4}$ then $\brak{\sec{2x} - \tan{2x}}$ equals
        
        \rightline{\color{magenta}\brak{1994}}
        \begin{enumerate}[label={(\alph*)}]
            \begin{multicols}{2}
                \item $\tan{\brak{x-\frac{\pi}{4}}}$
                \columnbreak
                \item $\tan{\brak{\frac{\pi}{4}-x}}$
                
            \end{multicols}

            \begin{multicols}{2}
                \item $\tan{\brak{x+\frac{\pi}{4}}}$ 
                \columnbreak
                \item $\tan^{2}{\brak{x+\frac{\pi}{4}}}$
            \end{multicols}
        \end{enumerate}

    %Question11
    \item Let n be a positive integer such that\\$\sin{\frac{\pi}{2n}} + \cos{\frac{\pi}{2n}} = \frac{\sqrt{n}}{2}$. Then
        
        \rightline{\color{magenta}\brak{1994}}
        \begin{enumerate}[label={(\alph*)}]
            \begin{multicols}{2}
                \item $6\le n\le8$
                \columnbreak
                \item $4<n\le8$
                
            \end{multicols}

            \begin{multicols}{2}
                \item $4\le n\le8$  
                \columnbreak
                \item $4<n<8$
            \end{multicols}
        \end{enumerate}

    %Question12
    \item If $\omega$ is an imaginary cube root of unity then the value of\\\\$\sin{\brak{\brak{\omega^{10} + \omega^{23}}\pi - \frac{\pi}{4}}}$ is
    
        \rightline{\color{magenta}\brak{1994}}
        \begin{enumerate}[label={(\alph*)}]
            \begin{multicols}{4}
                \item $-\frac{\sqrt{3}}{2}$
                \columnbreak
                \item $-\frac{1}{\sqrt{2}}$
                \columnbreak
                \item $-\frac{1}{\sqrt{2}}$
                \columnbreak
                \item $\frac{\sqrt{3}}{2}$
            \end{multicols}
        \end{enumerate}

    %Question13
    \item $3\brak{\sin{x} - \cos{x}}^4 + 6\brak{\sin{x} + \cos{x}}^4 + 4\brak{\sin^6{x}+\cos^6{x}} =$
        
        \rightline{\color{magenta}\brak{1995S}}
        \begin{enumerate}[label={(\alph*)}]
            \begin{multicols}{4}
                \item $11$
                \columnbreak
                \item $12$
                \columnbreak
                \item $13$
                \columnbreak
                \item $14$
            \end{multicols}
        \end{enumerate}   

    %Question14
    \item The general values of $\theta$ satisfying the equation\\$2sin^2{\theta}-3\sin{\theta}-2=0 is$
        
        \rightline{\color{magenta}\brak{1995S}}
        \begin{enumerate}[label={(\alph*)}]
            \begin{multicols}{2}
                \item $n\pi + \brak{-1}^n\frac{\pi}{6}$
                \columnbreak
                \item $n\pi + \brak{-1}^n\frac{\pi}{2}$
                
            \end{multicols}

            \begin{multicols}{2}
                \item $n\pi + \brak{-1}^n\frac{5\pi}{6}$ 
                \columnbreak
                \item $n\pi + \brak{-1}^n\frac{7\pi}{6}$
            \end{multicols}
        \end{enumerate}

    %Question15
    \item $\sec^2{\theta} = \frac{4xy}{\brak{x+y}^2}$ is true if and only if
        
        \rightline{\color{magenta}\brak{1996 - 1 Mark}}
        \begin{enumerate}[label={(\alph*)}]
            \begin{multicols}{2}
                \item $x+y=0$
                \columnbreak
                \item $x=y,x\neq0$
                
            \end{multicols}

            \begin{multicols}{2}
                \item $x=y$ 
                \columnbreak
                \item $x\neq0,y\neq0$
            \end{multicols}
        \end{enumerate}
        
    %Question16
    \item In a triangle $PQR$, $\angle R = \frac{\pi}{2}$. If $\tan{\frac{P}{2}}$ and $\tan{\frac{Q}{2}}$ are the roots of the equation $ax^2+bx+c=0 \;\brak{a\neq0}$ then
        
        \rightline{\color{magenta}\brak{1999 - 2 Marks}}
        \begin{enumerate}[label={(\alph*)}]
            \begin{multicols}{2}
                \item $a+b=c$
                \columnbreak
                \item $b+c=a$
                
            \end{multicols}

            \begin{multicols}{2}
                \item $a+c=b$ 
                \columnbreak
                \item $b=c$
            \end{multicols}
        \end{enumerate}

    %Question17
    \item Let $f\brak{\theta} = \sin{\theta}\brak{\sin{\theta} + \sin{3\theta}}$. Then $f\brak{\theta}$ is
        
        \rightline{\color{magenta}\brak{2000S}}
        \begin{enumerate}[label={(\alph*)}]
            \begin{multicols}{2}
                \item $\ge0$ only when $\theta\\\ge0$
                \columnbreak
                \item $\le0$ for all real $\theta$
            \end{multicols}

            \begin{multicols}{2}
                \item $\ge0$ for all real $\theta$
                \columnbreak
                \item $\le0$ only when $\theta\le0$
            \end{multicols}
        \end{enumerate}

    %Question18
    \item The number of distinct real roots of
    $$
    \begin{vmatrix}
        $$\sin{x}$$ & $$\cos{x}$$ & $$\cos{x}$$\\
        $$\cos{x}$$ & $$\sin{x}$$ & $$\cos{x}$$\\
        $$\cos{x}$$ & $$\cos{x}$$ & $$\sin{x}$$\\
    \end{vmatrix}
    $$
        \rightline{\color{magenta}\brak{2001S}}
        \begin{enumerate}[label={(\alph*)}]
            \begin{multicols}{4}
                \item $0$
                \columnbreak
                \item $2$
                \columnbreak
                \item $1$
                \columnbreak
                \item $3$
            \end{multicols}
        \end{enumerate}

    %Question19
    \item The maximum value of $\brak{\cos{\alpha_1}}\brak{\cos{\alpha_2}}\brak{\cos{\alpha_3}}...\brak{\cos{\alpha_n}}$ under the restrictions
    $$
    0\le\alpha_1,\alpha_2,...,\alpha_n\le\frac{\pi}{2} $$ and $$\brak{\cot{\alpha_1}}\brak{\cot{\alpha_2}}\brak{\cot{\alpha_3}}...\brak{\cot{\alpha_n}} = 1
    $$
        \rightline{\color{magenta}\brak{2001S}}
        \begin{enumerate}[label={(\alph*)}]
            \begin{multicols}{4}
                \item $\frac{1}{2^{\frac{n}{2}}}$
                \columnbreak
                \item $\frac{1}{2^{n}}$
                \columnbreak
                \item $\frac{1}{2n}$
                \columnbreak
                \item $1$
            \end{multicols}
        \end{enumerate}
\end{enumerate}
\end{document}

