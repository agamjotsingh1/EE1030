%iffalse
\let\negmedspace\undefined
\let\negthickspace\undefined
\documentclass[journal,12pt,twocolumn]{IEEEtran}
\usepackage{cite}
\usepackage{amsmath,amssymb,amsfonts,amsthm}
\usepackage{algorithmic}
\usepackage{graphicx}
\usepackage{textcomp}
\usepackage{xcolor}
\usepackage{txfonts}
\usepackage{listings}
\usepackage{enumitem}
\usepackage{mathtools}
\usepackage{gensymb}
\usepackage{comment}
\usepackage[breaklinks=true]{hyperref}
\usepackage{tkz-euclide} 
\usepackage{listings}
\usepackage{gvv}                                        
%\def\inputGnumericTable{}                                 
\usepackage[latin1]{inputenc}                                
\usepackage{color}                                            
\usepackage{array}                                            
\usepackage{longtable}                                       
\usepackage{calc}                                             
\usepackage{multirow}                                         
\usepackage{hhline}                                           
\usepackage{ifthen}                                           
\usepackage{lscape}
\usepackage{tabularx}
\usepackage{array}
\usepackage{float}
\usepackage{multicol}

\newtheorem{theorem}{Theorem}[section]
\newtheorem{problem}{Problem}
\newtheorem{proposition}{Proposition}[section]
\newtheorem{lemma}{Lemma}[section]
\newtheorem{corollary}[theorem]{Corollary}
\newtheorem{example}{Example}[section]
\newtheorem{definition}[problem]{Definition}
\newcommand{\BEQA}{\begin{eqnarray}}
\newcommand{\EEQA}{\end{eqnarray}}
\newcommand{\define}{\stackrel{\triangle}{=}}
\theoremstyle{remark}
\newtheorem{rem}{Remark}

% Marks the beginning of the document
\begin{document}
\bibliographystyle{IEEEtran}
\vspace{3cm}

\title{Assignment 2}
\author{EE24BTECH11002 - Agamjot Singh}
\maketitle
\newpage
\bigskip

\renewcommand{\thefigure}{\theenumi}
\renewcommand{\thetable}{\theenumi}

\begin{enumerate}
    \setcounter{enumi}{19}

    %Question20
    \item Let $a,b,c$ be such that $b(a+c)\neq0$ if

    		$\begin{vmatrix}
			a & a+1 & a-1\\
			b & b+1 & b-1\\
			c & c-1 & c+1\\
		\end{vmatrix}
		+
    		\begin{vmatrix}
			a+1 & b+1 & c-1\\
			a-1 & b-1 & c+1\\
			\brak{-1}^{n+2}a & \brak{-1}^{n+1}b & \brak{-1}^{n}c\\
		\end{vmatrix}
		= 0$,
		then the value of n is:

	\begin{enumerate}[label={(\alph*)}]
            \begin{multicols}{2}
                \item any even integer
                \columnbreak
		\item any odd integer 
            \end{multicols}

            \begin{multicols}{2}
		\item any integer 
                \columnbreak
		\item zero 
            \end{multicols}
	\end{enumerate}

    %Question21
    \item The number of $3\times3$ non-singular matrices with four entries as 1 and all othe entries as 0, is 
	\begin{enumerate}[label={(\alph*)}]
            \begin{multicols}{2}
                \item $5$ 
                \columnbreak
		\item $6$
            \end{multicols}

            \begin{multicols}{2}
		\item atleast $7$
                \columnbreak
		\item less than $4$ 
            \end{multicols}
	\end{enumerate}

    %Question22
    \item Let $A$ be a $2\times2$ matrix with non-zero entries and let $A^2 = I$, where $I$ is $2\times2$ identity matrix. Define 
	\newline
	Tr($A$) - sum of diagonal elements of $A$ and
	\newline
	$|A|$ - determinant of matrix $A$.
	\newline
	\textbf{Statement - 1:} Tr($A$) $= 0$.
	\newline
	\textbf{Statement - 2:} $|A|$ $= 1$

	\begin{enumerate}[label={(\alph*)}]
		\item Statement - 1 is true, Statement - 2 is true; Statement - 2 is \textbf{not} a correct explanation for Statement-1. 
	    	\item Statement - 1 is true, Statement - 2 is false. 
	    	\item Statement - 1 is false, Statement - 2 is true.
	    	\item Statement - 1 is true, Statement - 2 is true; Statement - 2 is a correct explanation for Statement-1. 
	\end{enumerate}

    %Question23
    \item Consider the system of linear equations; 
	\begin{align*}
		x_1 + 2x_2 + x_3 = 3\\
		2x_1 + 3x_2 + x_3 = 3\\
		3x_1 + 5x_2 + 2x_3 = 1
	\end{align*}
	\begin{enumerate}[label={(\alph*)}]
		\item exactly $3$ solutions
	    	\item a unique solution
	    	\item no solution
	    	\item infinite number of solutions
	\end{enumerate}

    %Question24
    \item The number of values of k for which the linear equations $4x + ky + 2z = 0$, $kx + 4y + z = 0$ and $2x + 2y + z=0$ posses a non zero solution is 
	\hfill{\brak{2011}}
        \begin{enumerate}[label={(\alph*)}]
            \begin{multicols}{4}
                \item $2$
                \columnbreak
                \item $1$
                \columnbreak
                \item zero
                \columnbreak
                \item $3$
            \end{multicols}
        \end{enumerate}

    %Question25
    \item Let $A$ and $B$ be two symmetrix matrices of order $3$.
	\newline
	\textbf{Statement - 1:} $A\brak{BA}$ and $\brak{AB}A$ are symmetric matrices. 
	\newline
	\textbf{Statement - 2:} $AB$ is symmetric matrix if matrix multiplication of A with B is commutative.

	\begin{enumerate}[label={(\alph*)}]
		\item Statement - 1 is true, Statement - 2 is true; Statement - 2 is \textbf{not} a correct explanation for Statement-1. 
	    	\item Statement - 1 is true, Statement - 2 is false. 
	    	\item Statement - 1 is false, Statement - 2 is true.
	    	\item Statement - 1 is true, Statement - 2 is true; Statement - 2 is a correct explanation for Statement-1. 
	\end{enumerate}

    %Question26
\item Let $ A = \begin{pmatrix} 1&0&0\\2&1&0\\3&2&1\end{pmatrix}$ If $u_1$ and $u_2$ are column matrices such that $Au_1 = \begin{pmatrix}1\\0\\0\end{pmatrix}$ and $Au_2 = \begin{pmatrix}1\\0\\0\end{pmatrix}$, then $u_1 + u_2$ is equal to:
	\hfill{\brak{2012}}
        \begin{enumerate}[label={(\alph*)}]
            \begin{multicols}{4}
	    	\item $\begin{pmatrix}-1\\1\\0\end{pmatrix}$ 
                \columnbreak
	    	\item $\begin{pmatrix}-1\\1\\-1\end{pmatrix}$ 
                \columnbreak
	    	\item $\begin{pmatrix}-1\\-1\\0\end{pmatrix}$ 
                \columnbreak
	    	\item $\begin{pmatrix}1\\-1\\-1\end{pmatrix}$ 
            \end{multicols}
        \end{enumerate}

    %Question27
	\item Let $P$ and $Q$ be $3\times3$ matrices $P\neq Q$. If $P^3=Q^3$ and $P^2Q=Q^2P$ then determinant of $\brak{P^2+Q^2}$ is equal to
	\hfill{\brak{2012}}
        \begin{enumerate}[label={(\alph*)}]
            \begin{multicols}{4}
                \item $-2$
                \columnbreak
                \item $1$
                \columnbreak
                \item $0$
                \columnbreak
                \item $-1$
            \end{multicols}
        \end{enumerate}

    %Question28
 	\item If $P = \begin{bmatrix}1&\alpha&3\\1&3^3\\2&4&4\end{bmatrix}$ is the adjoint of a $3\times3$ matrix $A$ and $|A| = 4$, then $\alpha$ is equa l to:
	\hfill{\brak{JEE M 2014}}
        \begin{enumerate}[label={(\alph*)}]
            \begin{multicols}{4}
                \item $4$
                \columnbreak
                \item $11$
                \columnbreak
                \item $5$
                \columnbreak
                \item $0$
            \end{multicols}
        \end{enumerate}

    %Question29
	\item If $\alpha,\beta,\neq 0$, and $f\brak{n} = \alpha^n + \beta^n$ and
		\newline
		\begin{align*}
			\begin{vmatrix}
				3 & 1+f\brak{1} & 1+f\brak{2}\\
				1+f\brak{1} & 1+f\brak{2} & 1+f\brak{3}\\
				1+f\brak{2} & 1+f\brak{3} & 1+f\brak{4}
			\end{vmatrix}
		\end{align*}

	\hfill{\brak{JEE M 2014}}
        \begin{enumerate}[label={(\alph*)}]
            \begin{multicols}{4}
                \item $4$
                \columnbreak
                \item $11$
                \columnbreak
                \item $5$
                \columnbreak
                \item $0$
            \end{multicols}
        \end{enumerate}

\end{enumerate}
\end{document}
